\chapter{Introduction}
\label{ch: Introduction}

\adjustmtc
\minitoc

\section{A section}
Here is an example of how a chapter page looks, without a minitoc (aka \textbf{mini} \textbf{t}able \textbf{o}f \textbf{c}ontents). 

\noindent This is a section.

\subsection{A subsection}
This is a subsection.

\begin{mccorrection}
Corrections can be inserted as so. 
\end{mccorrection}

Shorter corrections can be inserted \mccorrect{like so}\todo[size=\tiny]{Comments can also be done \textit{like this}. The FONTSIZE can be changed.}. 

\begin{figure}[!ht]
    \centering
    \includegraphics[width=0.2\textwidth]{figures/sample/laser.png}
    \caption{This is a figure. Courtesy of Suzanne Lim (CC BY-NC-ND 4.0).}
    \label{fig:laser}
\end{figure}

\section{Why \LaTeX{}?}
\subsection{Working At A Chapter At A Time}
A \textcolor{blue}{\textbf{major}} benefit of \LaTeX{} over Word is that you can write chapters, sections, etc as separate files, and then \say{include} or \say{input} them into the document being compiled (\verb|\include{}| \& \verb|\input{}|).\\
Unlike Word, you need to compile your document, which sounds daunting but as simple as the big green \say{Recompile} button or \say{ctr+s} on your keyboard.\\
The main file of this document is \verb|Oxford_Thesis.tex|. Which has \verb|\include{}| commands in each of the chapters being displayed right now.
If one wanted to not show a chapter in this final document, then all you have to do is comment it out with "\verb|%|".
You can use \verb|\input{}| within chapters to insert section or subsection files etc.\\\\
\textcolor{blue}{\textbf{Now}} you do not have to delete sections or chapters that you have previously written; rearranging chapters or sections is as simple as rearranging lines of code; you can compile the whole thesis as one final document less to worry with cross-referencing and citations.

\subsection{Citations \& Reference Managers}
Citations in \LaTeX{} do not slow down nearly as much as in Word in terms of adding and updating them. They are relatively easy to add with the \verb|\cite{}| command which quickly searches your reference file (in this case it's \verb|references.bib|), e.g., \verb|\cite{whitehead1927principia}| produces \cite{whitehead1927principia}.
\textcolor{blue}{\textbf{Additionally}}, with premium Overleaf, you can synch your Zotero, Mendeley, etc reference manager to your document, making it very easy. \textcolor{blue}{\textbf{Otherwise}} you would need to export out and update your reference file manually.

\subsection{Packages \& Commands}
The \LaTeX{} community is vast, and so, many clever people have worked to develop packages that you can easily add and use into your own work, adding extra functionality and quality-of-life features.
I (Robin) have written an extra chapter (\cref{ch: Examples}) just to highlight and give examples and usages of all of the useful packages and their commands that I used in my own thesis.\\
It will additionally help you to understand how to write \LaTeX{} too by looking at generated content and relating that to the source-code \verb|ch-02-Examples.tex| file.